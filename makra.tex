%%% Tento soubor obsahuje definice různých užitečných maker a prostředí %%%
%%% Další makra připisujte sem, ať nepřekáží v ostatních souborech.     %%%
%%% This file contains definitions of various useful macros and environments      %%%
%%% Assign additional macros here so that they do not interfere with other files. %%%

\usepackage[a-2u]{pdfx}     % výsledné PDF bude ve standardu PDF/A-2u
                            % resulting PDF will be in the PDF / A-2u standard

\usepackage{ifpdf}
\usepackage{ifxetex}
\usepackage{ifluatex}

%%% Nastavení pro použití samostatné bibliografické databáze.
%%% Settings for using a separate bibliographic database.
\usepackage[
   backend=biber
%  ,style=iso-authoryear
  ,style=iso-numeric
  ,sortlocale=cs_CZ
  ,alldates=iso
  ,bibencoding=UTF8
  ,maxnames=2
  ,maxbibnames=99
  %,block=ragged
]{biblatex}
\let\cite\parencite
\renewcommand*{\multinamedelim}{, \addspace}
\renewcommand*{\finalnamedelim}{\addspace a \addspace}

\bibliography{bibliography}

%% Přepneme na českou sazbu, fonty Latin Modern a kódování češtiny
\ifthenelse{\boolean{xetex}\OR\boolean{luatex}}
   { % use fontspec and OpenType fonts with utf8 engines
			\usepackage[english,slovak,czech]{babel}
			\usepackage[autostyle,english=british,czech=quotes]{csquotes}
			\usepackage{fontspec}
			\defaultfontfeatures{Ligatures=TeX,Scale=MatchLowercase}
   }
   {
			\usepackage[english,slovak,czech]{babel}
			\usepackage{lmodern}
			\usepackage[T1]{fontenc}
			\usepackage{textcomp}
			\usepackage[utf8]{inputenc}
			\usepackage[autostyle,english=british,czech=quotes]{csquotes}
	 }
\ifluatex
\makeatletter
\let\pdfstrcmp\pdf@strcmp
\makeatother
\fi

%%% Další užitečné balíčky (jsou součástí běžných distribucí LaTeXu)
\usepackage{amsmath}        % rozšíření pro sazbu matematiky / extension for math typesetting
\usepackage{amsfonts}       % matematické fonty / mathematical fonts
\usepackage{amssymb}        % symboly / symbols
\usepackage{amsthm}         % sazba vět, definic apod. / typesetting of sentences, definitions, etc.
\usepackage{bm}             % tučné symboly (příkaz \bm) / bold symbols (\bm command)
\usepackage{graphicx}       % vkládání obrázků / graphics inserting
\usepackage{listings}       % vylepšené prostředí pro strojové písmo / improved environment for source codes typesetting
\usepackage{fancyhdr}       % prostředí pohodlnější nastavení hlavy a paty stránek / environment for more comfortable adjustment of the head and foot of the pages
\usepackage{icomma}         % inteligetní čárka v matematickém módu / intelligent comma in math mode
\usepackage{dcolumn}        % lepší zarovnání sloupců v tabulkách / better alignment of columns in tables
\usepackage{booktabs}       % lepší vodorovné linky v tabulkách / better horizontal lines in tables
\makeatletter
\@ifpackageloaded{xcolor}{
   \@ifpackagewith{xcolor}{usenames}{}{\PassOptionsToPackage{usenames}{xcolor}}
  }{\usepackage[usenames]{xcolor}} % barevná sazba / color typesetting
\makeatother
\usepackage{multicol}       % práce s více sloupci na stránce / work with multiple columns on a page
\usepackage{caption}
\usepackage{enumitem}
\setlist[itemize]{noitemsep, topsep=0pt, partopsep=0pt}
\setlist[enumerate]{noitemsep, topsep=0pt, partopsep=0pt}
\setlist[description]{noitemsep, topsep=0pt, partopsep=0pt}

\usepackage{tocloft}
\setlength\cftparskip{0pt}
\setlength\cftbeforechapskip{1.5ex}
\setlength\cftfigindent{0pt}
\setlength\cfttabindent{0pt}
\setlength\cftbeforeloftitleskip{0pt}
\setlength\cftbeforelottitleskip{0pt}
\setlength\cftbeforetoctitleskip{0pt}
\renewcommand{\cftlottitlefont}{\Huge\bfseries\sffamily}
\renewcommand{\cftloftitlefont}{\Huge\bfseries\sffamily}
\renewcommand{\cfttoctitlefont}{\Huge\bfseries\sffamily}

% vyznaceni odstavcu
% differentiation of new paragraphs
\parindent=0pt
\parskip=11pt

% zakaz vdov a sirotku - jednoradkovych pocatku ci koncu odstavcu na prechodu mezi strankami
% Prohibition of widows and orphans - single-line beginning and end of paragraph at the transition between pages
\clubpenalty=1000
\widowpenalty=1000
\displaywidowpenalty=1000

% nastaveni radkovani
% setting of line spacing
\renewcommand{\baselinestretch}{1.20}

% nastaveni pro nadpisy - tucne a bezpatkove
% settings for headings - bold and sans serif
\usepackage{sectsty}    
\allsectionsfont{\sffamily}

% nastavení hlavy a paty stránek
% page head and foot settings
\makeatletter
\if@twoside%
    \fancypagestyle{fancyx}{%
			\fancyhf{}                                     
      \fancyhead[RE]{\rightmark}                  
      \fancyhead[LO]{\leftmark}                  
      \fancyfoot[RO,LE]{\thepage}                    
      \renewcommand{\headrulewidth}{.5pt}            
      \renewcommand{\footrulewidth}{.5pt}            
    }
    \fancypagestyle{plain}{%
			\fancyhf{}                                     
    	\fancyfoot[RO,LE]{\thepage}                    
    	\renewcommand{\headrulewidth}{0pt}             
    	\renewcommand{\footrulewidth}{0.5pt}
    }          
\else
    \fancypagestyle{fancyx}{%
			\fancyhf{}                                     
      \fancyhead[R]{\leftmark}                  
      \fancyfoot[R]{\thepage}                    
      \renewcommand{\headrulewidth}{.5pt}            
      \renewcommand{\footrulewidth}{.5pt}            
    }
    \fancypagestyle{plain}{%                       
    	\fancyhf{} % clear all header and footer fields
    	\fancyfoot[R]{\thepage}                    
    	\renewcommand{\headrulewidth}{0pt}             
    	\renewcommand{\footrulewidth}{0.5pt}
    }          
\fi
\renewcommand*{\cleardoublepage}{\clearpage\if@twoside \ifodd\c@page\else
	\hbox{}%
	\thispagestyle{empty}%
	\newpage%
	\if@twocolumn\hbox{}\newpage\fi\fi\fi
}
\makeatother

% Tato makra přesvědčují mírně ošklivým trikem LaTeX, aby hlavičky kapitol
% sázel příčetněji a nevynechával nad nimi spoustu místa. Směle ignorujte.
% These macros convince with a slightly ugly LaTeX trick to make chapter headers
% bet more sane and didn't miss a lot of space above them. Be boldly ignore it.
\makeatletter
\def\@makechapterhead#1{
  {\parindent \z@ \raggedright \sffamily
   \Huge\bfseries \thechapter. #1
   \par\nobreak
   \vskip 20\p@
}}
\def\@makeschapterhead#1{
  {\parindent \z@ \raggedright \sffamily
   \Huge\bfseries #1
   \par\nobreak
   \vskip 20\p@
}}
\makeatother

% Trochu volnější nastavení dělení slov, než je default.
% Slightly looser hyphenation setting than default.
\lefthyphenmin=2
\righthyphenmin=2

% Zapne černé "slimáky" na koncích řádků, které přetekly, abychom si jich lépe všimli.
% Turns on the black "snails" at the ends of the lines that overflowed to get us noticed them better.
\overfullrule=1mm

%% Balíček hyperref, kterým jdou vyrábět klikací odkazy v PDF,
%% ale hlavně ho používáme k uložení metadat do PDF (včetně obsahu).
%% Většinu nastavítek přednastaví balíček pdfx.
%% A hyperref package that can be used to produce clickable links in PDF,
%% but we mainly use it to store metadata in PDF (including content).
%% Most settings are preset by the pdfx package.
\hypersetup{unicode}
\hypersetup{breaklinks=true}
\hypersetup{hidelinks}

\renewcommand{\UrlBreaks}{\do\/\do\=\do\+\do\-\do\_\do\ \do\a\do\b\do\c\do\d%
\do\e\do\f\do\g\do\h\do\i\do\j\do\k\do\l\do\m\do\n\do\o\do\p\do\q\do\r\do\s%
\do\t\do\u\do\v\do\w\do\x\do\y\do\z\do\A\do\B\do\C\do\D\do\E\do\F\do\G\do\H%
\do\I\do\J\do\K\do\L\do\M\do\N\do\O\do\P\do\Q\do\R\do\S\do\T\do\U\do\V\do\W%
\do\X\do\Y\do\Z\do\1\do\2\do\3\do\4\do\5\do\6\do\7\do\8\do\9\do\0}

%%% Prostředí pro sazbu kódu, případně vstupu/výstupu počítačových
%%% programů. (Vyžaduje balíček listings -- fancy verbatim.)
%%% Environment for source code typesetting, or computer input/output
%%% programs. (Requires package listings - fancy verbatim.)
\lstnewenvironment{code}{\lstset{basicstyle=\small, frame=single}}{}

%%% User-defined balíčky
\usepackage{float}
\usepackage{tikz}
\usetikzlibrary{automata,positioning}
\usetikzlibrary{arrows.meta}
\usetikzlibrary{positioning}
\usetikzlibrary{shapes.geometric}

\usepackage{neuralnetwork}
\newcommand{\xin}[2]{$x_#2$}
\newcommand{\xout}[2]{$\hat x_#2$}

\usepackage[edges]{forest}
