\section{Diskuze}
\label{sec:vae_model_discussion}
V této kapitole byl prezentován jednoduchý variační autoenkodér, \textbf{jehož latentní prostor má pouze 2 dimenze}.
Což je užitečné pro vizualizace sloužící k prohledávání latentního prostoru.

Daleko lepších výsledků ale dosahují variační autoenkodéry, když dojde k zvýšení dimenze jejich latentního prostoru.

Představený model variačního autoenkodéru \textbf{je navržen tak, aby uměl pracovat s libovolnou sadou obrazových dat a rovněž uměl reprezentovat latentní prostor s vyšší dimenzí}.

Například, pro sestavení modelu variačního autoenkodéru pro generování realistických obličejů stačí pozměnit argumenty modelu následovně:

\lstinputlisting{code_snippets/vae_model_faces.py}


\textbf{Což poukazuje na skvělou flexibilitu rámce variačního autoenkodéru a stojí za detailnější prozkoumání možnosti generalizace}.