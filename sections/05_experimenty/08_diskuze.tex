\section{Diskuze}
\label{sec:vae_model_discussion}
V této kapitole byl prezentován jednoduchý variační autoenkodér, \textbf{jehož latentní prostor má pouze 2 dimenze}.
Představený model variačního autoenkodéru \textbf{je navržen tak, aby uměl pracovat s libovolnou sadou obrazových dat a rovněž uměl reprezentovat latentní prostor s vyšší dimenzí}.

Pro trénování modelu variačního autoenkodéru by tak stačilo uvést odlišnou konfiguraci konvolučních vrstev enkodéru a dekodéru, než kterou zachycuje \autoref{fig:vae_mnist_model}, a model by bez další změny fungoval.
\textbf{Což poukazuje na skvělou flexibilitu rámce variačního autoenkodéru a stojí za detailnější prozkoumání možnosti generalizace}.

Na evaluaci modelů s jinou datovou sadou již v této práci nezbyl prostor, ale zcela jistě by stály za další bádání. Zejména se zaměřením se na schopnost modelu variačního autoenkodéru generalizovat.

