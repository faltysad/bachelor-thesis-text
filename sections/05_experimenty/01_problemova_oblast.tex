\section{Vymezení problémové oblasti}
Cílem této kapitoly je vytvořit \textbf{ilustrační implementaci} modelu variačního autoenkodéru.

Jak napověděla \autoref{sec:applications_generative_modeling}, populární aplikací variačního autoenkodéru je úloha generativního modelování obrazových dat.
V rámci této kapitoly je tedy vytvořen \textbf{model variačního autoenkodéru pro generativní modelování ručně psaných číslic}.
Model je trénován na datové sadě MNSIT (viz \autoref{sec:mnist}).
Implementace jednotlivých částí modelu variačního autoenkodéru je propojena s teoretickými aspekty rámce VAE, které představila \autoref{chap:vae}.

Po natrénování modelu jsou prezentovány možnosti \textbf{generování zcela nových číslic} (viz \autoref{fig:mnist_example}) a je prozkoumán vzniklý latentní prostor modelu.

\textbf{Kompletní zdrojové kódy pro replikaci prezentovaných výsledků a vizualizací modelu jsou k dispozici v \autoref{app:vae_model_source_code}}.