\subsection{Robustní autoenkodér}
Robustní autoenkodér (\emph{robust autoencoder}) představuje další způsob, jakým se lze vypořádat s \emph{drobným šumem} ve vstupních datech, která má následně rekonstruovat.


Robustní autoenkodéry, narozdíl od denoising autoenkodéru (viz \autoref{sec:denoising_autoencoder}), využivájí alternativně definovanou ztŕatovou funkci, která je modifikována pro minimalizaci chyby rekonstrukce.
Obecně jsou ne-gaussovským šumem ovlivněny \textbf{méně než triviální mělké autoenkodéry} (viz \autoref{sec:shallow_autoncoder}).
Tato alternativní ztrátová funkce je založena na correntropii (\emph{correntropy}, lokalizovaná míra podobnosti), která je v \cite{Liu2006} definována následovně:

\begin{equation}
    \mathcal{L}_{MCC}(u, v) = -\sum_{k=1}^{d} \mathcal{K}_\sigma(u_k - v_k),
\end{equation}
kde
\begin{equation}
    \mathcal{K}_\sigma(\alpha) = \frac{1}{\sqrt[]{2\pi\sigma}}\exp(-\frac{\alpha^2}{2\sigma^2}),
\end{equation}
a $\sigma$ je parameter kernelu $\mathcal{K}$.

Correntropie měří hustotu pravdepodobnostni, že dvě \emph{události} jsou si rovny.
Correntropie je \textbf{výrazně méně ovlivněna odlehlými hodnotami}, než např. střední kvadratická chyba. \cite{Liu2006}
Robustní autoenkodér se snaží tuto míru maximalizovat, což intuitivně vede k \textbf{vyšší odolnosti} Robustního autoenkodéru \textbf{na ne-gaussovský šum} přítomný ve vstupních datech. \cite{Charte2018}
