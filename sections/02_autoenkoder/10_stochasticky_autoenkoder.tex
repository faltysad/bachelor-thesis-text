\subsection{Stochastický autoenkodér}
\label{sec:stochastic_autoencoder}
Struktura stochastického autoenkodéru se vůči obecné struktuře autoenkodéru (\autoref{fig:basic_autoencoder_structure}) liší reprezentací enkodér a dekodér modulů.

V stochastickém autoenkodéru jsou enkodér a dekodér moduly reprezentovány rozdělením pravděpodobností (nejedná se tedy pouze o funkce, ale moduly zahrnují i určitou míru šumu).
Výstup těchto modulů tedy obdržíme \textbf{výběrem z příslušného rozdělení pravděpodobnosti}. \cite{Goodfellow2016}

Mějme skrytou vrstvu (\emph{kód}) $\textbf{\emph{h}}$. Obecně má pro enkodér toto rozdělení podobu $p_{enkodér}(\textbf{\emph{h}}\mid\textbf{\emph{x}})$ a pro dekodér $p_{dekodér}(\textbf{\emph{x}}\mid\textbf{\emph{h}})$.
Modifikací základní struktury autoenkodéru (\autoref{fig:basic_autoencoder_structure}) tedy dostáváme:

\begin{figure}[h]
    \centering
    \begin{tikzpicture}[round/.style={circle, draw, minimum size=10mm, node distance=25mm, on grid}]
        \node[round](h){h};
        \node[round](x)[below left of=h]{x};
        \node[round](r)[below right of=h]{r};
        
        \draw[-Triangle] (x) -- node [left] {$p_{enkodér}(\textbf{\emph{h}}\mid\textbf{\emph{x}})$} (h);
        \draw[-Triangle] (h) -- node [right] {$p_{dekodér}(\textbf{\emph{x}}\mid\textbf{\emph{h}})$} (r);
    \end{tikzpicture}
    \caption{Obecná struktura autoenkodéru. Ze vstupu $\emph{x}$ je enkodérem vytvořen kód $\emph{h}$} (funkce $\emph{f}$). Tento kód je následně dekodérem přetaven na rekonstrukci $\emph{r}$ (funkce $\emph{g}$). Vlastní zpracování dle \textcite{Goodfellow2016}.
    \label{fig:stochastic_autoencoder_structure}
\end{figure}


V tradiční dopředné umělé neuronové síti \autoref{sec:feedforward_nn} je běžnou strategií pro návrh výstupní vrstvy definování (\emph{výstupního}) rozdělení pravděpodobnosti $p(\textbf{\emph{y}} \mid \textbf{\emph{x}})$. Pro návrh ztrátové funkce pak minimalizace záporného logaritmu věrohodnosti: $-\log p(\textbf{\emph{y}}\mid\textbf{\emph{x}})$.
V takové architektuře je $\textbf{\emph{x}}$ vektor vstupních dat a  $\textbf{\emph{y}}$ vektor cílových proměnných, které se umělá neuronová síť snaží předpovědět (např. štítků jednotlivých tříd). \cite{Goodfellow2016}

S architekturou autoenkodérů se však pojí jeden zásadní rozdíl. V autoenkodéru \textbf{je $\textbf{\emph{x}}$ jak vstupní, tak cílová proměnná}.

Dekodér pak lze interpretovat jako modul poskytující podmíněné rozdělení $p_{dekodér}(\textbf{\emph{x}} \mid \textbf{\emph{h}})$.
Autoenkodér lze trénovat minimalizací: $-\log p_{dekodér}(\textbf{\emph{x}} \mid \textbf{\emph{h}})$.
Konkrétní podoba ztrátové funkce se odvíjí od požadovaných vlastností autoenkodéru a od přesné podoby modulu dekodéru
(pokud hodnoty $\textbf{\emph{x}} \in \mathbb{R}$, pak jsou pro parametrizaci normálního rozdělení použity linéarní výstupní jednotky,
tedy: $-\log p_{dekodér}(\textbf{\emph{x}} \mid \textbf{\emph{h}})$ vrací \emph{střední kvadtratickou chybu}).

Stochastický autoenkodér tedy \textbf{generalizuje kódovací funkci} $f(x)$ na \textbf{kódovací rozdělení pravděpodobnosti} $p_{enkodér}(\textbf{\emph{h}}\mid\textbf{\emph{x}})$.

Enkodér a dekodér moduly stochastického autoenkodéru tedy lze považovat za \textbf{modely využívající latentní proměnné} (\emph{latent variable models}, viz \autoref{sec:latent_variable_models}).
Model využívající latentní proměnné značíme $p_{model}(\textbf{\emph{h}}, \textbf{\emph{x}})$. \cite{Goodfellow2016}

\textbf{Stochastický enkodér} je pak definován následovně \cite{Goodfellow2016}:
\begin{equation}
    p_{enkodér}(\textbf{\emph{h}}\mid\textbf{\emph{x}}) = p_{model}(\textbf{\emph{h}}, \textbf{\emph{x}})
\end{equation}

a \textbf{stochastický dekodér} následovně \cite{Goodfellow2016}:
\begin{equation}
    p_{dekodér}(\textbf{\emph{x}}\mid\textbf{\emph{h}}) = p_{model}(\textbf{\emph{x}}, \textbf{\emph{h}})
\end{equation}

Jak je popsáno v \autoref{chap:vae}, \textbf{stochastický autoenkodér se svým enkodér a dekodér modulem přibližuje k principu variačního autoenkodéru}.
Jedná se tedy o jakéhosi předchůdce variačního autoenkodéru, a řada představených principů platí i pro model variačního autoenkodéru. Nicméně stále existují vlastnosti, které zmiňuje \autoref{chap:vae}, kterých stochastický autoenkodér nenabývá.
Toto představuje nutnost pro zavedený autoenkodéru se spojitým latentním prostorem a stochastickými enkodér dekodér moduly, který nazýváme \textbf{variační autoenkodér} a detailně jej popisuje \autoref{chap:vae}.
