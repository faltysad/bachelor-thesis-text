\section{Řídký autoenkodér}
\label{sec:sparse_autoencoder}
Řídká reprezentace dat (\emph{sparsity}) ve strojovém učení znamená, že většina hodnot daného vzorku je nulová. 
Motivací pro řídkou reprezentaci dat ve strojovém učení je napodobení chování buněk v primární zrakové oblasti (\emph{V1}) mozku savců.
Konkrétně schopnosti odhalit a uložit efektivní kódovací strategie pozorovaných vjemů. 

Pro sestrojení Řídkého autoenkodérů je tedy nutné představit omezení (regularizační prvek) hodnot aktivací neuronů ve skryté (kódovací) vrstvě $\textbf{\emph{h}}$ (resp. počtu aktivních neuronů ve skrýte vrstvě).

Řídký autoenkodér (\emph{Sparse Autoecoder}) je Autoenkodér, jehož ztrátová funkce je rozšířena o penalizaci řídkosti kódovací vrstvy $\textbf{\emph{h}}$ (tzv. \emph{sparsity penalty}) vztahem $\Omega(\textbf{\emph{h}})$:

\begin{equation}
    L(\mathbf{x}, g(f(\mathbf{x}))) + \Omega(\textbf{\emph{h}}),
\end{equation}

kde $\Omega$ je \textbf{regularizační prvek}, jehož cílem je přiblížit hodnoty aktivací neuronů kódovací vrstvy k cílové hodnotě (a zabránit přeučení).
Chceme tak penalizovat neurony kódovací vrstvy, které se aktivují příliš často.

Běžně lze $\Omega$ stanovit následovně. Mějme Bernoulliho náhodnou proměnou $\emph{i}$ modelující aktivace neuronů skrýte (kódovací) vrstvy – můžou tedy nastat dva stavy: neuron skryté vrstvy je buď aktivován, nebo není aktivován.
Pro konkrétní vstup $\emph{x}$ dostaneme:

\begin{equation}
    \hat{p_i} = \frac{1}{|S|}\sum_{x \in S}^{}f_i(x),
\end{equation}

kde $ f = (f_1, f_2, \dots, f_c)$, $c$ je počet neuronů skryté (kódovací) vrstvy a $\hat{p_i}$ je průměrná aktivační hodnota neuronu skryté vrstvy (resp. střední hodnota příslušného Bernoulliho schématu).

Dále mějme $p$ jako cílové rozdělení aktivací.
Kullback-Leibnerova divergence mezi náhodnou proměnnou $i$ a $p$ pak udává rozdíl obou rozdělení:

\begin{equation}
    KL(p \parallel \hat{p_i}) = p \log \frac{p}{\hat{p_i}} + (1 - p) \log \frac{1 - p}{1 - \hat{p_i}}.
\end{equation}

Výsledný penalizační prvek $\Omega$ pro Řídký autoenkodér má tedy následující podobu:
\begin{equation}
    \Omega_{ŘAE}(W, b; S) = \sum_{i=1}^{c}KL(p \parallel \hat{p_i}),
\end{equation}

kde průměrná hodnota aktivací $\hat{p_i}$ závisí na parametrech enkodéru a množině trénovacích dat $S$. 

Přičtením tohoto penalizačního prvku ke ztrátové funkci (a následnou minimalizací celkové ztrátové funkce) je Autoenkodér nucen \textbf{omezit počet aktivních neuronů v skrýté (kódovací) vrstvě}.
V důsledku tohoto omezení pak každý neuron skryté vrstvy reprezentuje nějakou \textbf{salientní vlastnost} vstupních dat (a rovněž je zamezeno naučení pouhé identity \autoref{sec:identity}).
