\section{Od analýzy hlavních komponent po autoenkodér}
Máme-li lineární dekodér (Autoenkodér používá pouze lineární aktivační funkce) a jako ztrátová funkce $\mathcal{L}$ je použita \emph{střední kvadratická chyba},
pak se neúplný autoenkodér naučí stejný \emph{vektorový prostor}, který by byl výsledkem Analýzy hlavních komponent (viz \autoref{sec:pca}).
V tomto speciálním případě lze ukázat, že autoenkodér trénovaný na úloze kompresované reprezentace znalostí jako vedlejší efekt provedl Analýzu hlavních komponent. \cite{Baldi1989, Kamyshanska2013}

Důležitým důsledkem tohoto jevu je, že \textbf{autoenkodéry} s nelineární kódovací funkcí $f$
a nelineární dekódovací funkcí $g$ \textbf{jsou schopny učit se obecnější generalizaci}
než u Analýzy hlavních komponent. \cite{Goodfellow2016}

Na druhou stranu, je-li umělá neuronová síť autoenkodéru příliš vysoko-kapacitní,\footnote{Vysoko-kapacitní neuronové sítě, jsou sítě s vysokým počtem parametrů a vah, které jim umožňují učit se komplexním vztahům mezi daty. \cite[Kapitola 5]{Goodfellow2016}}
může se naučit pouze kopírovat vstup na výstupní vrstvě bez extrakce salientních vlastností (tedy identické zobrazení).