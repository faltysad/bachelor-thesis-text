\section{Ostatní typy autoenkoderů}
Na poli výzkumu strojového se autoenkodéry těší velkému úspěchu. Přirozeně tak existuje celá řada architektur a modifikací, které slouží k specifickým účelům.
V této kapitole byly představeny pouze ty architektury autoenkodéru, na kterých bude později stavět kapitola \autoref{chap:applications}.

V této sekci jsou stručně shrnuty \emph{ostatní} typy autoenkodérů, které nejsou pro předmět nadcházejících kapitol stěžejní, ale obecně se jim dostává vysoké míry využití.

\subsection{Adversariální autoenkodér}
Adversariální autoenkodér (\emph{adversarial autoencoder}) přináší koncept Generativních Adversariálních sítí \cite{Goodfellow2014} na pole autoenkodérů.
V adversariálním autoenkodéru je kód ($\textbf{\emph{h}}$) modelován uložením předchozího rozdělení pravděpodobnosti (\emph{prior}).
Následně je natrénován \emph{běžný} autoenkodér, současně se \emph{diksriminační} síť snaží odlišit kódy modelu od výběrů dat z předchozího rozdělení pravděpodobnosti.
Jelikož generátor (v tomto případě enkodér) je trénován k \emph{přelstění} diskriminátoru, kódy mají tendenci následovat uložené rozdělení pravděpodobnosti.
Tím pádem, z adversariálního autoenkodéru lze \textbf{provádět výběr zcela nových dat} (\emph{generovat nové vzorky}). \cite{Charte2018}

