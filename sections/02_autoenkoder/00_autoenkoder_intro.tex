Publikace \textcite{Kingma2014} stanovila základ pro model umělé neuronové sítě známý jako \textbf{variační autoenkodér}.
Variační autoenkodér se řadí mezi základní architektury pro úlohy generativního modelování a nabízí výtečnou výkonnost na široké škále úloh. \cite{Kingma2014, Kingma2019}

Pro představení teoretických aspektů variačního autoenkodéru je však nejprve nutné uvést \textbf{autoenkodér} – architekturu z 80. let minulého století,\footnote{Originální publikace dostupná z \textcite{Rumelhart1987}.} ze které variační autoenkodér vychází.

Konkrétně se tato kapitola věnuje charakteristikám jeho typů, které jsou postupně rozšířeny až k \textbf{variačnímu autoenkodéru}.
Představené charakteristiky jsou tím pádem nutně zachyceny ve vlastnostech \textbf{variačního autoenkodéru} a detailněji je popisuje \autoref{chap:vae}.

\autoref{sec:ae_princip} nejprve představuje \textbf{princip fungování autoenkodéru}, který je výchozím bodem pro návrh modelu umělé neuronové sítě variačního autoenkodéru.

\autoref{sec:undercomplete_autoencoder} zavádí \textbf{autoenkodér s neúplnou skrytou vrstvou}, jenž je inspirací pro mechanismus \textbf{komprimované reprezentace vstupných dat}, který je pro variační autoenkodér klíčový.

\autoref{sec:deep_ae} představuje princip \textbf{hlubokého autoenkodéru}, který následně při návrhu modelu variačního autoenkodéru popisuje \autoref{chap:experiments}.

\autoref{sec:sparse_autoencoder}, \autoref{sec:denoising_autoencoder} a \autoref{sec:contractive_autoencoder} prezentují možnosti využití \textbf{regularizace} pro dosažení konkrétní požadované vlastnosti modelu autoenkodéru.
Byť přesně tyto principy variační autoenkodér nevyužívá, slouží pro \textbf{akcentování důležitosti regularizace}, která je nedílnou součástí trénovací fáze variačního autoenkodéru.

\autoref{sec:stochastic_autoencoder} pak představuje \textbf{nejdůležitější část této kapitoly} – \textbf{stochastický autoenkodér}, jenž je prekurzorem pro model variačního autoenkodéru a sdílí spoustu nominálních vlastností.
\newpage