\section{Taxonomie autoenkodérů}
Bylo představeno několik tříd autoenkoderů (rozdělení dle navržení ANN sítě, rozdělení dle regularizačního prvku) a následně popsáno několik dalśích druhú AE.
Zde je jejich (nevyčerpávající) taxonomie.

\tikzset{
	basic/.style  = {draw, text width=5cm, rectangle},
	root/.style   = {basic, thin, align=center},
	level-2/.style = {basic, thin, align=center, text width=4.5cm},
	level-3/.style = {basic, thin, align=center, text width=3.75cm}
}
\begin{figure}[H]
	\centering
	\begin{tikzpicture}[
			level 1/.style={sibling distance=13em, level distance=6em},
			edge from parent/.style={->,solid,black,thick,sloped,draw}, 
			edge from parent path={(\tikzparentnode.south) -- (\tikzchildnode.north)},
		>=latex, node distance=1.5cm, edge from parent fork down]
    
    \node[root] {\textbf{Autoenkodér}}
        child {node[level-2] (c1) {\textbf{Dle kompozice skryté vrstvy}}}
        child {node[level-2] (c2) {\textbf{Dle formulace regularizačního prvku}}}
        child {node[level-2] (c3) {\textbf{Dle reprezentace enkodér dekodér modulů}}};

    \begin{scope}[every node/.style={level-3}]
        \node [below of = c1, xshift=40pt] (c11) {Autoenkodér s neúplnou skrytou vrstvou};
        \node [below of = c11] (c12) {Autoenkodér s rozšířenou skrytou vrstvou};
        \node [below of = c12] (c13) {Hluboký autoenkodér};
        
        \node [below of = c2, xshift=40pt] (c21) {Řídký autoenkodér};
        \node [below of = c21] (c22) {Denoising autoenkodér};
        \node [below of = c22] (c23) {Contractive autoenkodér};
        \node [below of = c23] (c24) {Stochastický enkodér-dekodér};
        
        \node [below of = c3, xshift=40pt] (c31) {Funkce};
        \node [below of = c31] (c32) {Rozdělení pravděpodobnosti};
    \end{scope}

    \foreach \value in {1,...,3}
        \draw[->] (c1.195) -- (c1\value.west);

    \foreach \value in {1,...,4}
        \draw[->] (c2.195) -- (c2\value.west);
      
    \foreach \value in {1,...,2}
        \draw[->] (c3.195) -- (c3\value.west);
		  
	\end{tikzpicture}
	\caption{Autoenkodéry rozděleny dle charakteristik zpracování kódovací vrstvy}
	\label{fig:autoencoder_taxonomy}
\end{figure}