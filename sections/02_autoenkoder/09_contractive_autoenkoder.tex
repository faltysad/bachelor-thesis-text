\subsection{Contractive autoenkodér}
\label{sec:contractive_autoencoder}
Přehnaná citlivost na \emph{drobné rozdíly} ve vstupních datech by mohla vést k architektuře Autoenkodéru, která pro velmi podobné vstupy generuje odlišné kódy.

Contractive autoenkodér (\emph{CAE}), je Autoenkodér, který je při trénování omezen regularizačním prvkem,
který vynucuje aby derivace kódů ve vztahu k jejich vstupu byly co možná nejmenší.
Tedy \textbf{dva \emph{podobné} vstupy musí mít vzájemně \emph{podobné} kódy}.
Přesněji je dosaženo lokální invariance na přípustně malé změny vstupních dat. \cite{Rifai2012}

Citlivost na \emph{drobné rozdíly} ve vstupních datech lze měřit pomocí Frobeniovy maticové normy $\lVert \cdot \rVert_F$ Jacobiho matice enkodéru ($J_f$):
\begin{equation}
    \lVert J_f(x) \rVert^2_F = \sum_{j=1}^{d}\sum_{i=1}^{c} \left( \frac{\partial f_i}{\partial x_j} (x) \right) ^2 .
\end{equation}

Čím vyšší je tato hodnota, tím více bude kód nestabilní s ohledem na \emph{drobné rozdíly} ve vstupních datech.
Z této metriky je následně sestaven \textbf{regularizační prvek} který je připočten k hodnotě ztrátové funkce Contractive Autoenkodéru:

\begin{equation}
    \Omega_{CAE} (W, b, S) = \sum_{x \in S}^{} \lVert J_f(x) \rVert^2_F .
\end{equation}

Výsledkem je tedy Autoenkodér, jehož dva (lokálně) \emph{podobné} vstupy musejí mít i \emph{podobný} kód. \cite{Charte2018}

Z contractive autoenkodéru lze rovněž vzorkovat nové výstupy.
Jakobián (\emph{Jacobiho determinant}) enkodéru je (jako \emph{drobný šum}) přičten ke kódu vstupu.
Takto modifikovaný kód je poté dekodérem přetaven na výstup a dostáváme nový vzorek dat. \cite{Goodfellow2016}

Tento princip se začíná podobat fungování vzorkování ve variačním autoenkodéru, kde je rovněž možné (byť z jiného důvodu) vzorkovat data, která nebyla součástí trénovací množiny.
Vzorkování z modelu variačního autoenkodéru popisuje \autoref{chap:vae} a implementuje \autoref{chap:experiments}.