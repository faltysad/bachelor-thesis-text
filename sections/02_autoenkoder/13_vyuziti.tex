\section{Využítí Autoenkodéru}
Představené architektury autoenkodérů nabízejí řadu využití, mezi které se řadí:
\begin{itemize}
    \item Mapování vysokorozměrných dat do 2D pro vizualizaci
    \item Učení se abstraktních vlastností o vstupních datech bez učitele pro následné využití v úlohách učení s učitelem
    \item Komprese dat
\end{itemize}

nicméně naráží na zásadní problém nespojitosti naučených manifoldů – tedy (s výjimkou stochastického přístupu, viz \autoref{sec:stochastic_autoencoder}) nemají příliš dobrý výkon v generativních úlohách.
Aplikacím úloh pozorování v latentním prostoru je věnována celá \autoref{chap:applications}.
Pro jejich realizaci je však nutné představit architekturu \textbf{variačního autoenkodéru} (viz \autoref{chap:vae}), která staví na dosud popsaných konceptech.