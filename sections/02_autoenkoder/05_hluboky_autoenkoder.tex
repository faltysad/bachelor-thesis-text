\subsection{Hluboký autoenkodér}
\label{sec:deep_ae}
\autoref{sec:undercomplete_autoencoder} představila autoenkodér s jednovrstvým enkodérem a s jednovrstvým dekodérem.
Existují však i autoenkodéry, jejichž enkodér a dekodér moduly jsou vícevrstvé, tedy mají hloubku vyšší než jedna.

Hluboký autoenkodér (\emph{deep} autoenkodér), je autoenkodér s netriviální hloubkou skryté vrstvy (\emph{kódu}) $\textbf{\emph{h}}$. \cite{Geron2019}

\begin{figure}[H]
    \centering
    \begin{neuralnetwork}[height=6]
        \tikzstyle{input neuron}=[neuron, circle, draw=black, fill=white];
        \tikzstyle{hidden neuron}=[neuron, circle, draw=black, fill=white];
        \tikzstyle{output neuron}=[neuron, circle, draw=black, fill=white];
      
        \inputlayer[count=6, bias=false, text=\xin]
        \hiddenlayer[count=4, bias=false]
        \linklayers
        \hiddenlayer[count=2, bias=false]
        \linklayers
        \hiddenlayer[count=4, bias=false]
        \linklayers
        \outputlayer[count=6, text=\xout]
        \linklayers
      \end{neuralnetwork}
    \caption{Schéma umělé neuronové sítě hlubokého autoenkodéru s neúplnou skrytou vrstvou.}
    \label{fig:stacked_autoencoder}
\end{figure}

U mělkých enkodérů, které jsou rovněž dopřednou sítí, neexistuje možnost představit libovolná omezení a regularizační prvky (například řídkost \emph{kódu} $\textbf{\emph{h}}$ – viz \autoref{sec:sparse_autoencoder}).
A tedy nelze zamezit problému naučení pouhé identity (viz \autoref{sec:identity}). Na rozdíl od mělkých autoenkodérů (viz \autoref{sec:shallow_autoncoder}),
však mohou hluboké autoenkodéry (\emph{libovolně přesně}) aproximovat jakékoliv mapování vstupu na kód (\emph{opět s předpokladem dostatečného počtu neuronů skryté vrstvy}). \cite{Goodfellow2016, Charte2018}

S netriviální hloubkou se rovněž pojí významná redukce výpočetních nákladů spojených s reprezentací některých funkcí. 
V důsledku možnosti efektivnější reprezentace takových funkcí dochází i k zmenšení nároků na velikost množiny trénovacích dat.

\subsection{Stacked autoenkodér}
\label{sec:stacked_autoencoder}
Běžnou strategií pro trénování Hlubokého autoenkodéru je hladově předtrénovat (\emph{greedy pretraining}) model individuálním natrénováním většího počtu Mělkých autoenkodérů (představených v sekci \autoref{sec:shallow_autoncoder}), které jsou následně vloženy za sebe.
Takto složený Autoenkodér nazýváme Stacked autoenkodér. \cite{Geron2019}