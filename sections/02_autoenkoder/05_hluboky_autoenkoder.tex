\section{Hlubuký autoenkodér}
V sekci \autoref{sec:undercomplete_autoencoder} a \autoref{sec:overcomplete_autoencoder} byly představeny Autonkodéry s jednovrstvým enkodérém a s jednovrstvým dekodérem.
Existují však i Autoenkodéry, jejichž enkodér a dekodér moduly jsou vícevrstvé, tedy mají hloubku vyšší než jedna.

Hluboký autoenkodér (\emph{Deep} autoenkodér), je Autoenkodér s netriviální hlobkou skryté vrstvy (\emph{kód}) $\textbf{\emph{h}}$.

\begin{figure}[H]
    \centering
    \begin{neuralnetwork}[height=6]
        \tikzstyle{input neuron}=[neuron, circle, draw=black, fill=white];
        \tikzstyle{hidden neuron}=[neuron, circle, draw=black, fill=white];
        \tikzstyle{output neuron}=[neuron, circle, draw=black, fill=white];
      
        \inputlayer[count=6, bias=false, text=\xin]
        \hiddenlayer[count=4, bias=false]
        \linklayers
        \hiddenlayer[count=2, bias=false]
        \linklayers
        \hiddenlayer[count=4, bias=false]
        \linklayers
        \outputlayer[count=6, text=\xout]
        \linklayers
      \end{neuralnetwork}
    \caption{Deep Autoenkodér.}
    \label{fig:stacked_autoencoder}
\end{figure}

Z netriviální hloubky dopředné umělé neuronové sítě plyne \autoref{sec:universal_approximation_theorem},
který garantuje, že dopředná umělá neuronová síť s alespoň jednou skrytou vrstvou dokáže aproximovat (\emph{libovolně přesně}) jakoukoliv funkci (za předpokladu dostatečného počtu neuron skryté vrstvy).

Nicméně u mělkých enkodérů, které jsou rovněž dopřednou sítí, neexistuje možnost představit libovolná omezení a regularizační prvky (například řídkost \emph{kódu} $\textbf{\emph{h}}$ – viz \autoref{sec:sparse_autoencoder}).
A tedy nelze zamezit problému naučiení pouhé identity \autoref{sec:identity}. Narozdíl od mělkých autoenkoderů \autoref{sec:shallow_autoncoder},
však mohou Hluboké autoenkodéry (\emph{libovolně přesně}) aproximovat jakékoliv mapování vstupu na kód (\emph{opět s předpokladem dostatečného počtu neuronů skryté vrstvy}).

S netriviální hloubkou se rovněž pojí významná redukce výpočetních nákladů spojených s reprezentací některých funkcí. 
V důsledku možnosti efektivnější reprezentace takových funkcí dochazí i k zmenšení nároků na velikost množiny trénovacích dat.

\subsection{Stacked autoenkodér}
\label{sec:stacked_autoencoder}
Běžnou strategií pro trénování Hlubokého autoenkodéru je hladově předtrénovat (\emph{greedy pretraining}) model individuálním natrénováním většího počtu Mělkých autoenkodérů (představených v sekci \autoref{sec:shallow_autoncoder}), které jsou následně vloženy za sebe.
Takto složený Autoenkodér nazýváme Stacked autoenkodér.