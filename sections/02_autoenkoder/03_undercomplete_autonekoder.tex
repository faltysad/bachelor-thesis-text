\subsection{Autoenkodér s neúplnou skrytou vrstvou}
\label{sec:undercomplete_autoencoder}
Autoenkodér s neúplnou skrytou vrstvou (\emph{undercomplete autoencoder}) je autoenkodér, jehož dimenze kódu (\emph{h}) je menší, než dimenze vstupu
\footnote{Naopak autoenkodér, jehož počet neuronů ve skryté vrstvě je větší, než počet neuronů vstupní (a výstupní) vrstvy se nazývá \textbf{autoenkodér s rozšířenou skrytou vrstvou}. \cite{Charte2018}}.
Tuto skrytou vrstvu \emph{h} nazýváme \textbf{bottleneck}. Bottleneck je způsob, kterým se autoenkodér s neúplnou skrytou vrstvou učí \textbf{komprimované reprezentaci znalostí}. 
V důsledku bottleneck vrstvy je autoenkodér nucen zachytit pouze výrazné rysy trénovacích dat, které následně budou použity pro rekonstrukci. \cite{Goodfellow2016}

Trénovací proces autoenkodéru s neúplnou skrytou vrstvou je popsán jako minimalizace ztrátové funkce $\mathcal{L}(\mathbf{x}, g(f(\mathbf{x})))$,
kde ztrátová funkce $\mathcal{L}$, penalizuje $g(f(\mathbf{x}))$ za \emph{rozdílnost} vůči $\mathbf{x}$ (např. \emph{střední kvadratickou chybou}). \cite{Charte2018}

\begin{figure}[H]
    \centering
    \begin{neuralnetwork}[height=4]
        \tikzstyle{input neuron}=[neuron, circle, draw=black, fill=white];
        \tikzstyle{hidden neuron}=[neuron, circle, draw=black, fill=white];
        \tikzstyle{output neuron}=[neuron, circle, draw=black, fill=white];
      
        \inputlayer[count=4, bias=false, title=Enkodér, text=\xin]
      
      
        \hiddenlayer[count=2, bias=false, title=Kód $\emph{h}$]
        \linklayers
      
        \outputlayer[count=4, title=Dekodér, text=\xout]
        \linklayers
      
      \end{neuralnetwork}
    \caption{Jednoduchá architektura umělé neuronové sítě Autoenkodéru s neúplnou skrytou vrstvou. Skrytá vrstva představuje \emph{bottleneck}.}
    \label{fig:autoencoder_bottleneck}
\end{figure}