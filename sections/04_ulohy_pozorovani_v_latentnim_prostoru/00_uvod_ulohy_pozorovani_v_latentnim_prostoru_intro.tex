Hlavním cílem této práce je popsat princip fungování variačního autoenkodéru.
Avšak ještě před samotným zavedením jeho teoretických aspektů, tato kapitola prezentuje \textbf{motivaci k jeho použití}.


\textbf{Variační autoenkodér} (dále jen VAE) je generativní model využívající latentních proměnných.

Generativní modely využívající latentních proměnných umožňují transformaci vstupních dat
do \emph{jednodušších} a lépe interpretovatelných prostorů\footnote{Tyto prostory se nazývají \emph{latentní} a je jim věnována \autoref{sec:latent_variable_model}.}.
A tím pádem umožňují taková data lépe prozkoumat a porozumět jim. \cite{Kingma2019}

Na pozorování těchto prostorů jsou založeny aplikace, které tato kapitola prezentuje a je z něj odvozen i název celé práce.

Tato kapitola je zaměřena na ty nejprominentnější aplikace variačního autoenkodéru, které lze rozčlenit do následujících kategorií:

\begin{itemize}
    \item \textbf{Generativní modelování obrazových dat}: \autoref{sec:applications_generative_modeling}, \autoref{sec:applications_image_resynthesis}, \autoref{sec:applications_image_coloring}, \autoref{sec:applications_image_reconstruction}.
    \item \textbf{Syntéza přirozeného jazyka}: \autoref{sec:applications_language_synthesis}. 
    \item \textbf{Syntéza pseudo-dat}: \autoref{sec:applications_chemical_pseudo_data_synthesis} a \autoref{sec:applications_astronomy_pseudo_data_synthesis}.
\end{itemize}

Jednotlivé sekce pouze \textbf{ilustrují} využití rámce VAE pro konkrétní úlohu dané problémové oblasti a prezentují výsledky dosažené v dostupných publikacích.
Pro detailní popis architektury využitého modelu a jeho evaluaci jsou v každé sekci odkázány původní publikace.

Způsob realizace a teorii modelu variačního autoenkodéru, použitého pro dosažení těchto výsledků, pak prezentují navazující kapitoly práce.
Konkrétně:
\begin{itemize}
    \item \autoref{chap:prereqs} uvádí východiska variačního autoenkodéru ukotvené v oblasti strojového učení.
    \item \autoref{chap:autoencoder} popisuje architekturu autoenkodéru, předchůdce variačního autoenkodéru z 80. let minulého století.
    \item \autoref{chap:vae} zavádí a postupně představuje kompletní teorii a principy fungování variačního autoenkodéru.
    \item \autoref{chap:experiments} implementuje model variačního autoenkodéru pro řešení generativní úlohy obrazových dat MNIST.
\end{itemize}