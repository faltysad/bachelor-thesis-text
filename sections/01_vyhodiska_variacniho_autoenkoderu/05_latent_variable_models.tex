\section{Hluboký model využívající latentních proměnných}
\label{sec:latent_variable_models}
Model využívající latentních proměnných (\emph{latent variable model}) se značí $\emph{p}_\theta(\mathbf{x}, \mathbf{z})$.

Jsou-li rozdělení pravděpodobností tohoto modelu parametrizovány neuronovou sítí,
nazývá se \textbf{hluboký model využívající latentních proměnných} (\emph{deep latent variable model}, dále jen \emph{DLVM}).

DLVM může být podmíněný vůči libovolnému kontextu $y$, tedy $\emph{p}_{\theta}(\mathbf{x}, \mathbf{z}|\mathbf{y})$.
Důležitou vlastností (a výhodou) DLVM je, že marginální rozdělení $\emph{p}_\theta$ může být velmi komplexní (tedy obsahovat téměř jakékoliv závislosti).
Tato expresivní vlastnost činí DLVM atraktivní pro \textbf{aproximaci složitých skrytých rozdělení pravděpodobnosti} $\emph{p}^*(\mathbf{x})$. \cite{Kingma2019}

Jedním z nejjednodušších (a nejznámějších) DLVM modelů je faktorizace s následující strukturou \cite{Kingma2019}:
\begin{equation}
    \emph{p}_\theta(\mathbf{x}, \mathbf{z}) = \emph{p}_\theta(\mathbf{z})\emph{p}_\theta(\mathbf{x}\mid\mathbf{z}),
\end{equation}

kde $\emph{p}_\theta(\mathbf{z})$ a $\emph{p}_\theta(\mathbf{x}\mid\mathbf{z})$ jsou předem dány.

Dle \textcite{Kingma2019} lze modely variačního autoenkodéru zařadit mezi hluboké modely využívající latentních proměnných.
\subsection{Latentní proměnné}
Latentní proměnné jsou proměnné, které jsou součástí modelu, ale nejsou přímo pozorovatelné (a tím pádem nejsou součástí vstupních dat).
Pro označení takových proměnných je použito $\mathbf{z}$. U latentních proměnných se předpokládá jejich vliv na hodnotu cílové proměnné (resp. cílových proměnných). \cite{Kingma2019}
