\section{Pravděpodobnostní modely}
Jak napovídá definice v \autoref{sec:machine_learning_algorithm}, existuje mnoho druhů úloh $T$, ke kterým je možné využít algoritmy strojového učení.
Tato práce se zabývá především úlohami v pravděpodobnostním kontextu. Oblast strojového učení, která tento typ úloh řeší se nazývá \textbf{pravděpodobnostní modelování}. 

Pravděpodobnostní modely strojového učení pracují s cílovou proměnnou (třídou) jako s \textbf{náhodnou proměnnou, která podléhá nějakému rozdělení pravděpodobnosti}.
Toto rozdělení popisuje váženou množinu hodnot, kterých může náhodná proměnná nabýt. \cite{Murphy2022}

Pro nutnost adopce pravděpodobnostních modelů hovoří dva hlavní důvody
– jedná se o \textbf{optimální přístup} pro \textbf{rozhodování s faktorem neurčitosti}
a poté, pravděpodobností modelování je \emph{přirozené} pro řadu inženýrských disciplín (například stochastická optimalizace, operační výzkum, ekonometrie, informační teorie a další). \cite{Murphy2022}

