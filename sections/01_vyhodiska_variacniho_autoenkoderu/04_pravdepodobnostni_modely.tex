\section{Pravděpodobnostní modely}
\label{sec:probabilstic_models}
Tato práce se zabývá především úlohami v pravděpodobnostním kontextu.
Oblast strojového učení, která řeší modelování úloh stochastického charakteru se nazývá \textbf{pravděpodobnostní modelování}.
Variační autoenkodér je řazen do třídy pravděpodobnostních modelů. \cite{Kingma2019}.

Pravděpodobnostní modely strojového učení pracují s cílovou proměnnou (třídou) jako s \textbf{náhodnou proměnnou, která podléhá nějakému rozdělení pravděpodobnosti}.
Toto rozdělení popisuje váženou množinu hodnot, kterých může náhodná proměnná nabýt. \cite{Murphy2022}

Jako dvě hlavní východiska existence pravděpodobnostních modelů \textcite{Murphy2022} prezentuje:

\begin{itemize}
    \item Jedná se o \textbf{optimální přístup} pro \textbf{rozhodování s faktorem neurčitosti}.
    \item Pravděpodobnostní modelování je \emph{přirozené} pro řadu inženýrských disciplín (například stochastická optimalizace, operační výzkum, ekonometrie, informační teorie a další).
\end{itemize}

