\section{Redukce dimenzionality}
\label{sec:dimensionality_reduction}

Do oblasti redukce dimenzionality patří celá řada technik pro práci s vysokodimenzionálními daty.
Cílem této oblasti, na rozdíl od regresních problémů, není predikovat hodnotu cílové proměnné – ale porozumět tvaru dat, se kterými pracuje.
Typickou úlohou redukce dimenzionality dat je sestrojit \textbf{nízkodimenzionální reprezentaci}, která zachytí \emph{většinu významu} původní, vysokodimenzionální, reprezentace.
Tento jev nazýváme hledáním \textbf{salientních vlastností} původní sady dat. \cite{Phillips2021}

Redukci dimenzionality lze chápat jako úlohu učení bez učitele, ve které se algoritmus strojového učení učí mapování z vysokodimenzionálního prostoru $\textbf{\emph{x}} \in \mathbb{R}^D$ do nízkodimenzionálního latentního prostoru $\textbf{\emph{z}} \in \mathbb{R}^L$. \cite{Murphy2022}

\subsection{\textquote{The curse of dimensionality}}
S rostoucím počtem vstupních proměnných exponenciálně roste počet vzorků dat nutný pro \emph{libovolně přesnou} aproximaci dané funkce.
V důsledku je tedy s rostoucí dimenzí vstupních dat značně degradováno chování většiny algoritmů (strojového učení).
Tento problém je známý pod termínem \textbf{curse of dimensionality}. \cite{Bellman1957}

I proto došlo ke vzniku oblasti zvané \emph{feature engineering}.
Feature engineering je oblast, která se, mimo jiné, zabývá disciplínou selekce vlastností dat, které budou použity při trénování modelu.
Pro automatizovanou selekci vlastností existuje celá řada technik.
Výběr podprostoru, který \emph{nejlépe} reprezentuje vlastnosti původních dat je NP-těžký kombinatorický problém (\emph{exhaustive search through all the subsets of features}).
Tyto techniky dokonce často vyhodnocují každou vstupní proměnnou nezávisle, což může vést ke zkresleným závěrům o jejich významnosti – naopak je běžné, že proměnné začínají vykazovat určitou míru významnosti \textbf{až při vzájemném využití}. \cite{Stanczyk2015}

Výše stanovené důvody vedly k emergenci další disciplíny, a to \textbf{extrakce vlastností}, o níž pojednává \autoref{sec:feature_extraction}. 

\subsection{Extrakce vlastností}
\label{sec:feature_extraction}
Cílem extrakce vlastností \emph{feature extraction} je najít reprezentaci vstupních dat, která je vhodná pro algoritmus strojového učení, který se chystáme využít (jelikož původní reprezentace může být z mnoha důvodů nevhodná – například vysokodimenzionální).
Typicky tak musí dojít k redukci dimenzionality vstupních dat. \cite{Liu1998}

K extrakci nových vlastností lze dojít mnoha způsoby.
Existují techniky založené na hledání lineárních kombinací původních vstupních vlastností, například Analýza hlavních komponent (\emph{PCA Analýza}) nebo Lineární diskriminační analýza (\emph{LDA Analýza}).

Pro nelineární redukci dimenzionality lze využít technik tzv. \emph{manifold learningu}, viz \cite[kap. 5.11.3.]{Goodfellow2016}. 


\subsection{Analýza hlavních komponent (PCA)}
\label{sec:pca}
Jednou z nejrozšířenějších metod pro redukci dimenzionality je PCA analýza (\emph{principal component analysis}).

V principu se jedná o nalezení lineární a ortogonální projekce vysokodimenzionálních dat $\textbf{\emph{x}} \in \mathbb{R}^D$ do nízkodimenzionálního podprostoru  $\textbf{\emph{z}} \in \mathbb{R}^L$,
tak aby tato nízkodimenzionální reprezentace byla \emph{dobrou aproximací} původních dat.
Provedeme-li projekci (resp. \textbf{kódování}) $\textbf{\emph{x}}$, získáme $\mathbf{z} = \mathbf{W^{\top}} \textbf{\emph{x}}$ (funkci kódování označme $e$),
a následně \textbf{dekódujeme} $\textbf{\emph{z}}$ abychom získali $\hat{\textbf{\emph{x}}} = \mathbf{W}\textbf{\emph{z}}$ (funkci dekódování označme $d$),
pak chceme aby $\hat{\textbf{\emph{x}}}$ bylo co možná nejblíže $\textbf{\emph{x}}$ (měřeno pomocí $\ell_2$ vzdálenosti). \cite{Murphy2022}

Konkrétně můžeme definovat následující \textbf{chybu rekonstrukce} \cite{Murphy2022}:
\begin{equation}
    \mathcal{L}(\mathbf{W}) \triangleq \frac{1}{N} \| \textbf{\emph{x}} - d(e(\textbf{\emph{x}}_n; \mathbf{W}); \mathbf{W})\|^2_2
\end{equation}

kde $e$ a $d$ jsou lineární funkce (\emph{mappings}).

Na PCA analýzu tak lze pohlížet jako na algoritmus učení bez učitele, který se učí reprezentaci vstupních dat. Tato reprezentace má dvě kritéria \cite{Goodfellow2016}:

\begin{itemize}
    \item PCA se učí reprezentaci vstupních dat, která má menší dimenzi než původní vstupní reprezentace.
    \item PCA se učí reprezentaci, jejíž elementy mezi sebou nemají žádnou lineární korelaci.
\end{itemize}
