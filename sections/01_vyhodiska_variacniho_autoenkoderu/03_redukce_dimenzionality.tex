\section{Redukce dimenzionality}
\label{sec:dimensionality_reduction}

Do oblasti redukce dimenzionality patří celá řada technik pro práci s vysokodimenzionálními daty.
Cílem této oblasti, na rozdíl od regresních problémů, není predikovat hodnotu cílové proměnné – ale porozumět tvaru dat, se kterými pracuje.
Typickou úlohou redukce dimenzionality dat je sestrojit \textbf{nízkodimenzionální reprezentaci}, která zachytí \emph{většinu významu} původní, vysokodimenzionální, reprezentace.
Tento jev nazýváme hledáním \textbf{salientních vlastnostní} původní sady dat. \cite{Phillips2021}

\subsection{The curse of dimensionality}
S rostoucím počtem vstupních proměnných exponenciálně roste počet vzorků dat nutný pro \emph{libovolně přesnou} aproximaci dané funkce.
V důsledku je tedy s rostoucí dimenzí vstupních dat značně degradováno chování většiny algoritmů (strojového učení).
Tento problém je známý pod termínem \textbf{curse of dimensionality}. \cite{Bellman1957}

I proto došlo k vzniku oblasti zvané \emph{feature engineering}.
Feature engineering je oblast, která se, mimo jiné, zabývá disciplínou selekce vlastností dat, které budou použity při trénování modelu.
Pro automatizovanou selekci vlastností existuje celá řada technik.
Výběr podprostoru, který \emph{nejlépe} reprezentuje vlastnosti původních dat je NP-těžký kombinatorický problém (\emph{exhaustive search through all the subsets of features}).
Tyto techniky dokonce často vyhodnocují každou vstupní proměnnou nezávisle, což může vést ke zkresleným závěrům o jejich významnosti – naopak je běžné, že proměnné začínají vykazovat určitou míru významnosti \textbf{až při vzájemném využití}. \cite{Stanczyk2015}

Výše stanovené důvody vedly k emergenci další disciplíny, a to \textbf{extrakce vlastností}, která je pro předmět této práce patřičně důležitější.

\subsection{Extrakce vlastností}
Cílem extrakce vlastností \emph{feature extraction} je najít reprezentaci vstupních dat, která je vhodná pro algoritmus strojového učení, který se chystáme využít (jelikož původní reprezentace může být z mnoha důvodů nevhodná – například vysokodimenzionální).
Typicky tak musí dojít k redukci dimenzionality vstupních dat. \cite{Liu1998}

K extrakci nových vlastností lze dojít mnoha způsoby.
Existují techniky založené na hledání lineárních kombinací původních vstupních vlastností, například Analýza hlavních komponent (\emph{PCA Analýza}) nebo Lineární diskriminační analýza (\emph{LDA Analýza}).

Pro nelineární redukci dimenzionality lze využít technik tzv. manifold learningu.


\subsection{Analýza hlavních komponent}
\label{sec:pca}

\subsection{Manifold learning}