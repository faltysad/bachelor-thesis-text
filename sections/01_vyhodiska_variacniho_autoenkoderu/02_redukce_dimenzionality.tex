\section{Redukce dimenzionality}
\label{sec:dimensionality_reduction}

Do oblasti reducke dimenzionality patří celá řada technik pro práci s vysokodimenzionálními daty.
Cílem této oblasti, narozdíl od regresních problémů, není predikovat hodnotu cílové proměnné – ale porozumět tvaru dat, se kterými pracuje.
Typickou úlohou redukce dimenzionality dat je sestrojit \textbf{nízkodimenzionální reprezentaci}, která zachytí \emph{většinu významu} původní, vysokodimenzionální, reprezentace.
Tento jev nazýváme hledáním \textbf{salientních vlastnostní} původní sady dat. \cite{Phillips2021}

\subsection{The curse of dimensionality}
S rostoucím počtem vstupních vlastností je chování většiny algoritmů (strojového učení) značně degradováno.
Tento problém je známý pod termínem \textbf{curse of dimensionality}. \cite{Bellman1957}

I to vedlo ke vzniku oblasti zvané \emph{feature engineering}.
Tedy selekci vlastností dat, které při trénování modelu budou využity.
Selekce pouze \emph{nejlepšího} podprostoru vstupních vlastností dat je NP-těžký kombinatorický problém.

\subsection{Analýza hlavních komponent}
\label{sec:pca}