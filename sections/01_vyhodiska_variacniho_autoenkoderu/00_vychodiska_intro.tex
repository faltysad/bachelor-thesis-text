Kapitola je věnována představení principů z oblasti strojového učení,
které jsou \textbf{předpokladem} pro zavedení teorie autoenkodéru a \textbf{variačního autoenkodéru}.

Variační autoenkodér je \textbf{modelem} v oblasti \textbf{strojového učení}, proto je nejprve vymezeno strojového učení (\autoref{sec:machine_learning}), algoritmus strojového učení (\autoref{sec:machine_learning_algorithm}) a princip modelu v kontextu strojového učení.

Kapitola také definuje relevantní třídy algoritmů strojového učení – \textbf{učení s učitelem} (\autoref{sec:supervised_learning}) a \textbf{učení bez učitele} (\autoref{sec:unsupervised_learning}), jejichž charakteristiky jsou využity při návrhu zpracování dat uvnitř modelu variačního autoenkodéru a návrhu vrstev jeho dílčích modulů.

Modely variačního autoenkodéru mají z podstaty svého návrhu predispozici k řešení úloh z určité podmnožiny existujících problémových oblastí.
\autoref{sec:no_free_lunch} obecně zdůvodňuje \textbf{nutnost návrhu modelu variačního autoenkodéru za sledovaným účelem}.

Model variačního autoenkodéru a jeho latentní reprezentace nabývá určitých vlastností, které pozdější kapitoly práce představují.
Za účelem formulace způsobu dosažení těchto vlastností je popsán princip \textbf{regularizace} (\autoref{sec:regularization}).

Jak v pozdější fázi práce ukazuje \autoref{chap:vae}, model variačního autoenkodéru (i prostého autoenkodéru) je složen ze dvou modulů – \textbf{enkodéru} a \textbf{dekodéru}.
Tyto moduly jsou ve fázi práce věnované experimentu s modelem variačního autoenkodéru \textbf{implementovány formou umělých neuronových sítí}. Proto \autoref{sec:neural_network} definuje jejich principy, historický vývoj a schopnost univerzální reprezentace.

Návrh variačního autoenkodéru je založen na principu \textbf{redukce dimenzionality} vstupních dat.
Stručnému definování tohoto principu a příslušných technik, ze kterých variační autoenkodér vychází, je věnována \autoref{sec:dimensionality_reduction}.

Variační autoenkodér je \textbf{pravděpodobnostním modelem} a využívá \textbf{latentních proměnných}.
Jejich představení je věnována \autoref{sec:probabilstic_models} a \autoref{sec:latent_variable_model}.

Velmi důležitou součástí trénovacího procesu variačního autoenkodéru je \textbf{KL divergence}, kterou zavádí \autoref{sec:kl_divergence}.

Tato kapitola je výchozím bodem a \emph{seznamem základních pojmů} pro zbytek práce.
\textbf{Čtenáři, který je se zvýrazněnými oblastmi seznámen, je doporučeno kapitolu přeskočit}.