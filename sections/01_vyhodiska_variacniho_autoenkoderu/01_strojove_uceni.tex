\section{Strojové učení}
Strojové učení je podoblast umělé inteligence zabývající se algoritmy a technikami, které počítačovým systémům umožňují \emph{učit se} z dat.
\begin{definition}[Strojové učení]
    \label{def:machine_learning}
    \textquote{\emph{Machine Learning is the field of study that gives computers the ability to learn without being explicitly programmed}.} \cite{Samuel1967}
\end{definition}
\subsection{Algoritmus strojového učení}
\label{sec:machine_learning_algorithm}
Definici algoritmu strojového učení výstižně shrnuje \textcite[str. 2]{Mitchell1997}:
\begin{definition}[Algoritmus strojového učení]
    \label{def:machine_learning_algorithm}
    \textquote{\emph{A computer program is said to learn from experience $E$ with respect to some class of tasks $T$ and performance measure $P$, if its performance at tasks in $T$, as measured by $P$, improves with experience $E$.}}
\end{definition}
Algoritmy strojového učení lze obecně rozdělit na čtyři třídy – učení s učitelem (\emph{supervised learning}),
učení bez učitele (\emph{unsupervised learning}), kombinaci učení s učitelem a učení bez učitele (\emph{semi-supervised learning}) a posilovaného učení (\emph{reinforcement learning}).

Toto dělení vychází ze zkušenosti $E$, resp. míry, do jaké má algoritmus strojového učení \textbf{povoleno} interagovat s datovou sadou (resp. jakou možnost taková datová sada nabízí). \cite{Goodfellow2016}

Aplikováním algoritmu strojového učení na konkrétní úlohu, a jeho následným \textbf{trénováním} za účelem získání určitých charakteristik, vzniká \textbf{model strojového učení}.
Takový model se \textbf{učí} opakovaným procesem, kdy na základě nějaké \textbf{chyby} (forma zpětné vazby) \textbf{optimalizuje ztrátovou funkci}, která je definována specificky k řešené úloze.
Tento opakovaný proces nazýváme \textbf{trénování}, a detailněji jej interpretuje \autoref{sec:neural_network_architecture}.

Pro předmět této práce budou blíže představeny pouze oblasti \textbf{učení s učitelem} a \textbf{učení bez učitele}.
\subsection{Učení s učitelem}
\label{sec:supervised_learning}
Zkušenost $E$ algoritmů učení bez učitele vychází z datové sady, která může mít celou řadu vlastností a zároveň je \textbf{předem známá klasifikace každého datového bodu do definovaných tříd}.
Na základě této asociace je natrénovaný algoritmus schopen provést přiřazení dosud neznámého objektu do jedné z tříd definovaných v trénovací sadě dat. \cite{Chollet2017}

Učení s učitelem zahrnuje pozorování několika příkladů náhodného vektoru $\mathbf{x}$ a asociované hodnoty (či vektoru) $\mathbf{y}$.
Cílem úlohy učení se s učitelem pak je naučit se predikovat $\mathbf{y}$ z $\mathbf{x}$, často na základě odhadu $p(\mathbf{y}\mid\mathbf{x})$, kde $p$ je podmíněná pravděpodobnost výskytu jevu $y$, pakliže se vyskytl jev $x$. \cite{Goodfellow2016}

Samotný termín \emph{učení s učitelem} vychází ze situace, kdy je cílová třída $\mathbf{y}$ poskytnuta jakýmsi instruktorem či učitelem, který systému strojového učení ukazuje očekávané chování. \cite{Goodfellow2016}
\subsection{Učení bez učitele}
\label{sec:unsupervised_learning}
Dle \autoref{def:machine_learning} a \autoref{def:machine_learning_algorithm} lze říci, že zkušenost $E$ algoritmů učení bez učitele vychází z datové sady, která může mít celou řadu vlastností.
Cílem trénování algoritmů učení bez učitele je \textbf{naučit se užitečné a charakteristické vlastnosti o struktuře vstupní datové sady}.

Lze uvažovat i tzv. \textbf{hluboké učení} bez učitele, kdy cílem algoritmu je naučit se celé rozdělení pravděpodobnosti, které generuje původní datovou sadu (ať už explicitně – např. odhad hustoty rozdělení pravděpodobnosti, či implicitně – např. úlohy syntézy dat a odstranění šumu). \cite{Chollet2017}

Obecně lze říct, že učení bez učitele zahrnuje pozorování několika příkladů náhodného vektoru $\mathbf{x}$ na základě kterého se snaží implicitně či explicitně \emph{naučit} rozdělení pravděpodobnosti $p(\mathbf{x})$, případně \emph{užitečné vlastnosti} tohoto pravděpodobnostního rozdělení.

Na rozdíl od \autoref{sec:supervised_learning}, název \emph{učení bez učitele} napovídá, že v procesu učení není zapojen žádný \emph{instruktor} či \emph{učitel}, a tak systém strojového učení sám musí vyvodit smysl a užitečné vlastnosti předložené datové sady. \cite{Goodfellow2016}
\subsection{\textquote{No free lunch theorem} pro strojové učení}
\label{sec:no_free_lunch}
\textcite{Vapnik2000} naznačuje, že by algoritmus strojového učení mohl mít schopnost generalizace i z konečné množiny trénovacích dat. 
Toto tvrzení je ale v rozporu s principy \emph{klasické} logiky – indukce obecných pravidel \cite{Hume1978} z omezeného vzorku dat je dle těchto principů nevalidní.
Chceme-li provést indukci obecného pravidla, které popisuje každý prvek množiny, musíme mít k dispozici informaci o každém prvku z množiny.  \cite{Goodfellow2016}

Pro logické vyvrácení takto naučené generalizace stačí byť jeden vzorek, který je s tímto pravidlem v nesouladu a nebyl součástí trénovací množiny (tzv. \emph{Black swan paradox}, viz \textcite{Taleb2008}).

Oblast strojového učení se tomuto paradoxu z části vyhýbá tím, že pracuje pouze s \textbf{pravděpodobnostními pravidly} (oproti zcela určitým pravidlům jako v logické indukci).
Algoritmy strojového učení hledají pravidla, která jsou tzv. \emph{probably approximately correct}. \cite{Valiant1984}

Ani tento trik však kompletně neřeší představený problém. Zjednodušeně, \textbf{"No free lunch theorem"} pro strojové učení tvrdí, že každý klasifikační algoritmus má v průměru, skrze všechny distribuce generující data, \textbf{stejnou chybovost} při klasifikování dosud nepozorovaných datových bodů. \cite{Wolpert1996}
Jinými slovy, \textbf{žádný algoritmus strojového učení není univerzálně lepší než kterýkoliv jiný algoritmus strojového učení}.

Toto tvrzení je pravdivé až při zohlednění \emph{všech možných distribucí generujících data} a za předpokladu, že jim je přidělena stejná váha \cite{Hibbard2009} – tedy v teoretické rovině.
Lze pozorovat, že v praktických aplikacích je možné navrhnout algoritmus, který si v určitých distribucích vede v průměru lépe než ostatní algoritmy. \cite{Goodfellow2016}

Tato práce představuje variační autoenkodér – architekturu umělé neuronové sítě, která se těší dobrým výsledkům ve vybraných úlohách učení bez učitele, představených v \autoref{chap:applications}, ale v důsledku má i své nedostatky (představené v \autoref{chap:vae}) v ostatních třídách úloh.

\subsection{Regularizace}
\label{sec:regularization}
\textquote{No free lunch theorem} pro strojové učení (představený v \autoref{sec:no_free_lunch}) implikuje nutnost návrhu algoritmu strojového učení pro konkrétní úlohu, chceme-li aby jeho výkonnost byla vyšší než výkonnost ostatních algoritmů v průměru.
Toho lze docílit \emph{zabudováním} určité sady preferencí přímo do algoritmu strojového učení (předpokladem je, že tyto preference jsou v souladu s cílovým problémem, který se algoritmus snaží řešit). \cite{Goodfellow2016}


Chování a výkonnost algoritmu strojového učení lze dle \textcite{Goodfellow2016} ovlivnit omezením jeho prostoru hypotéz následovně:
\begin{itemize}
    \item V prostoru hypotéz algoritmu lze vyjádřit preferenci jednoho řešení před druhým. To znamená, že obě takové funkce budou přípustné, ale jedna z nich má preferenci.
    \item Funkci (nebo množinu funkcí) lze z prostoru hypotéz kompletně vyřadit (resp. vyjádřit nekonečně velkou míru neupřednostnění takové funkce).
\end{itemize}

Obecně můžeme regularizovat model přičtením \emph{trestu} k jeho ztrátové funkci. Tento trest nazýváme \textbf{regularizační prvek} (\emph{regularizer}) a značíme jej $\Omega$.

Regularizaci je pak definována jako:
\begin{definition}[Regularizace]
    \label{def:regularization}
    \textquote{\emph{Regularization is any modification we make to a learning algorithm that is intended to reduce its generalization error but not its training error. Regularization is one of the central concerns of the field of machine learning, rivaled in its importance only by optimization}.} \cite{Goodfellow2016}
\end{definition}

Formu regularizace je nutné pečlivě zvolit s ohledem na typ úlohy, který má algoritmus za cíl řešit.

\textbf{Variačního autoenkodér} (a určité typ\ prostého autoenkodéru, tzv. regularizované autoenkodéry, viz \autoref{chap:autoencoder}) \textbf{regularizace} dle \autoref{def:regularization} \textbf{využívá}.

Jak v pozdějším textu představuje \autoref{chap:vae}, ztrátová funkce variačního autoenkodéru je regularizována za účelem prevence přeučení a směřování modelů variačního autoenkodéru směrem k určité množině chtěných vlastností (například regularizace vůči normovanému normálnímu rozdělení). \cite{Kingma2014}







