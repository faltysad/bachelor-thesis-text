\section{Umělá neuronová síť}
\label{neural_network}
Umělá neuronová síť je model strojového učení inspirovaný přírodou.
Zdá se být intuitivní, že chceme-li napodobit lidskou inteligenci, měli bychom se pro inspiraci podívat na architekturu lidského mozku.
V průběhu času se však konstrukce umělých neuronových sítí začala jejich přírodnímu protějšku podstatně vzdalovat.
Řada architektur umělých neuronových sítí tvoří biologicky nerealistický model, byť z této původní myšlenky vychází (stejně tak jako letadla vycházejí z přírodního vzoru létajících ptáků, pohybem svými křídly se ve skutečnosti ve vzduchu neodrážejí). \cite{Geron2019}

Představení kompletních principů umělých neuronových sítí není východiskem variačního autoenkodéru, nýbrž svým obsahem pokrývají několik monografií – pro úvod například \cite{Chollet2017}, \cite{Geron2019}.
V této sekci tedy budou představeny pouze stěžejní techniky využívané autoenkodéry.
\subsection{Architektura umělé neuronové sítě}
Pojem \textbf{architektura} rozumíme \textbf{celkovou strukturu} umělé neuronové sítě – počet neuronů a způsob jakým jsou tyto neurony mezi sebou propojeny, jejich aktivační funkce a podobně.

Většina umělých neuronových sítí je organizována do skupin neuronů zvaných \textbf{vrstvy}.
Architektury umělých neuronových sítí poté uspořádávají tyto vrstvy do zřetězené struktury, kde každá vrstva je funkcí předcházející vrstvy. \cite{Goodfellow2016}

V takové struktuře je první vrstva dána následovně:
\begin{equation}
    \textbf{\emph{h}}^{(1)} = g^{(1)} \left( \mathbf{W}^{(1)\top}\textbf{\emph{x}}+\textbf{\emph{b}}^{(1)} \right),
\end{equation}
a druhá vrstva je dána následovně:
\begin{equation}
    \textbf{\emph{h}}^{(2)} = g^{(2)} \left( \mathbf{W}^{(2)\top}\textbf{\emph{h}}^{(1)}+\textbf{\emph{b}}^{(2)} \right).
\end{equation}

Mezi hlavní architektonická rozhodnutí při návrhu umělé neuronové sítě patří volba \textbf{hloubky sítě} a \textbf{šířky každé z vrstev}.
\subsection{Anatomie umělé neuronové sítě}
Proces trénování umělé neuronové sítě z pravidla zahrnuje následující objekty \cite{Chollet2017}:
\begin{itemize}
    \item \emph{Vrstvy} ze kterých je následně složena \emph{síť} (resp. \emph{model})
    \item \emph{Vstupní data} (a případně jejich \emph{cílové třídy})
    \item \emph{Ztrátová funkce}, která slouží jako signál zpětné vazby použití pro učení modelu
    \item \emph{Optimizér}, který modifikuje parametry umělé neuronové sítě (např. váhy)
\end{itemize}

\subsection{Dopředná umělá neuronová síť}
\label{sec:feedforward_nn}
Dopředná umělá neuronová síť je velmi podstatná pro návrh (hlubokých) modelů strojového učení.
Cílem dopředné umělé neuronové sítě je aproximovat nějakou funkci $f*$.
\footnote{Například pro klasifikátor, $y=f*(x)$ mapuje vstup $x$ do kategorie $y$. Převzato z \cite{Goodfellow2016}.}
Dopředná umělá neuronová síť definuje mapovací funkci $\textbf{\emph{y}} = \emph{f}(\textbf{\emph{x}};\boldsymbol{\theta})$ a učí se hodnotu parametrů $\boldsymbol{\theta}$, jejichž výsledky je nejlepší aproximace cílové funkce. \cite{Goodfellow2016}

\textbf{Dopředné} umělé neuronové \textbf{sítě} nazýváme:
\begin{itemize}
    \item \emph{Dopředné}, jelikož v této architektuře neexistují žádné zpětnovazební propojení, které by sloužily jako vstup zpět pro sebe sama
    \footnote{Je-li dopředná umělé neuronové sítě obohacena o zpětnovazební propojení se sebou sama, nazýváme ji \textbf{Rekurentní neuronová síť}.}.
    \item \emph{Sítě}, jelikož jsou typicky reprezentovány kompozicí více různých funkcí.
    Model takové sítě je asociován s orientovaným acyklickým grafem, který popisuje jakým způsobem je tato kompozice tvořena
    Mějme tři funkce $f^{(1)}$, $f^{(2)}$, $f^{(3)}$ které jsou zřetězeny následovně: $f(\textbf{\emph{x}}) = f^{(3)}(f^{(2)}(f^{(1)}(\textbf{\emph{x}})))$.
    Takto zřetězené struktury nazýváme \textbf{vrstvy} umělé neuronové sítě. Délku tohoto zřetězení nazýváme \textbf{hloubkou} modelu – od tud \emph{hluboké} neuronové sítě (umělé neuronové sítě s 2 a více skrytými vrstvami). \cite{Goodfellow2016}
    \footnote{V tomto případě je $f^{(1)}$ \textbf{první vrstva}, $f^{(2)}$ je \textbf{druhá vrstva} a $f^{(3)}$ \textbf{třetí vrstva}.}.
\end{itemize}

\subsection{Perceptron}
\label{sec:perceptron}
Jedna z nejjednodušších architektur umělé neuronové sítě (a zároveň \emph{model umělého neuronu}) představena v \cite{Rosenblatt1957} inspirována principy Hebbovského učení \cite{Hebb1949}.
Perceptron přichází s důležitým principem \textbf{numerických hodnot vstupů, výstupů} (oproti pouhým binárním hodnotám) – tzv \emph{linear treshold unit, LTU} a \textbf{vah} mezi jednotlivými neurony. 
Perceptron byl později kritizován \cite{Minsky1969} za jeho neschopnost řešit triviální problémy (např. \emph{XOR} klasifikace), což eventuálně vedlo ke krátkodobé stagnaci konekcionismu.

Jak se ale ukázalo, některé z těchto limitací lze vyřešit uspořádáním více Perceptronů za sebe do vrstev. \cite{Rumelhart1987}

Takto uspořádaná umělá neuronová síť se nazývá Vícevrstvý Perceptron.

\subsection{Vícevrstvý Perceptron a Algoritmus zpětné propagace}
\label{sec:multilayer_perceptron}
Vícevrstvý Perceptron je tvořen umělou neuronovou sítí, která se skládá z jedné vstupní vrstvy, \textbf{jedné nebo více skrytých vrstev} LTU jednotek, a jedné výstupní vrstvy rovněž složené z LTU jednotek.
Součástí každé vrstvy (vyjma výstupní) je tzv. \textbf{bias} neuron který je \textbf{plně propojený} s další vrstvou sítě. \cite{Geron2019}

Pokud má umělá neuronová síť dvě a více skrytých vrstev, nazýváme ji \textbf{hlubokou neuronouvou sítí}.

Jak již víme, Vícevrstvý Perceptron adresuje limitace Perceptronu (viz \autoref{sec:perceptron}).
Článek \cite{Rumelhart1987} ale představil i další revoluční myšlenku – algoritmus \textbf{zpětné propagace}
\footnote{Algoritmus zpětné propagace byl ve skutečnosti nezávisle objeven více výzkumníky, počínaje \cite{Werbos1974}. Převzato z \cite{Geron2019}.} (\emph{backpropagation}).

Algoritmus zpětné propagace lze velmi zjednodušeně interpretovat následovně: Pro každou trénovací instanci je vypočten výstup každého jejího neuronu každé jednotlivé vrstvy.
Poté je změřena výstupní chyba celé sítě (například střední kvadratická chyba, MSE) a vypočten podílem, jakým každý neuron poslední skryté vrstvy přispěl k hodnotě každého neuronu výstupní vrstvy.
Následně je stejným způsobem měřeno, jak moc byla ovlivněna hodnota jednotlivých neuronů poslední skryté vrstvy hodnotami neuronů předchozí skryté vrstvy – a podobný proces se opakuje než algoritmus dosáhne vstupní vrstvy. \cite{Geron2019}

Průchod algoritmu sítí lze intuitivně popsat následovně: Pro každou trénovací instanci algoritmus zpětné propagace nejprve učiní predikci cílové hodnoty – tzv. \emph{forward pass}).
Poté změří chybu této predikce. Následně zpětně projde každou vrstvu sítě za účelem změření míry přispění k této chybě každým propojením (a její váhou) – tzv. \emph{reverse pass}.
Závěrem algoritmus \emph{lehce} upraví váhy jednotlivých propojení za účelem snížení chyby – tzv. \emph{Gradient Descent step}. \cite{Geron2019}

Pro správný chod algoritmu autoři \cite{Rumelhart1987} do architektury Vícevrstvého Perceptronu zanesli další zásadní změnu – jako tzv. \textbf{aktivační funkci} využili logistickou regresi (zcela běžné je využití i dalších aktivačních funkcí, například ReLU).

\subsection{Univerzální aproximační teorém}
\label{sec:universal_approximation_theorem}
Univerzální aproximační teorém\footnote{Universal approximation theorem} \cite{Hornik1989}, \cite{Cybenko1989} tvrdí, že dopředná umělá neuronová síť (s alespoň jednou skrytou vrstvou a \emph{potlačující} aktivační funkcí – např. logistická (sigmoida)) je schopna aproximovat libovolnou (borelovsky měřitelnou) funkci s \emph{libovolnou mírou přesnosti}\footnote{Derivacemi dopředné umělé neuronové sítě lze stejně tak aproximovat derivace cílové funkce \cite{Hornik1990}.}.

Důsledkem Univerzálního aproximačního teorému je, že dostatečně velký \textbf{Vícevrstvý perceptron zvládne \emph{reprezentovat} libovolnou funkci}.
Ale neexistuje zde jistota, že se algoritmus strojového učení bude schopen takovou funkci vždy \emph{naučit}  \cite{Goodfellow2016}:

\begin{itemize}
    \item Zvolený optimalizér algoritmu strojového učení použitý pro trénování nemusí být schopen najít parametry umělé neuronové sítě, které korespondují s cílovou funkcí.
    \item Algoritmus nemusí vybrat správnou funkci v důsledku přeučení
\end{itemize}

Dopředné umělé neuronové sítě tak nabízejí univerzální systém pro \textbf{reprezentaci} libovolné funkce. \textbf{Pro libovolnou funkci existuje dopředná umělá neuronová síť, která ji aproximuje}
\footnote{S ohledem na \autoref{sec:no_free_lunch} neexistuje žádný univerzální způsob pro prozkoumání trénovací množiny dat a zvolení funkce, která bude libovolně přesně generalizovat na instance dat, které nejsou součástí této trénovací množiny.}.