\section{Tří různé pohledy na interpretaci naučeného modelu}
Jak bylo ukázáno, proces učení variačních autoenkodérů je efektivně řešitelný problém (\emph{tractable}).

Při učení je optimalizován $\log P(X)$ skrze celou množinu vstupních dat $D$. Nicméně, není optimalizován \emph{přesně} $\log P(X)$, ale pouze jeho odhad.

Tato sekce slouží pro odkrytí mechanismů, které se skutečně na pozadí účelové funkce variačního autoenkodéru dějí.
Nabízí tři různé pohledy, kterými může být \autoref{eq:vae_elbo} interpretována:
\begin{enumerate}
    \item Velikost chyby, kterou způsobuje současná optimalizace $\mathcal{D}_{KL}(q_\phi(z\mid x^{(i)})\parallel p_\theta(z))$ dodatečně vedle optimalizace $\log p_\theta(x^{(i)})$.
    \item Interpretace v kontextu informační teorie a propojení s dalšími přístupy založenými na Minimal Description Length.
    \item VAE a regularizační prvek. Existuje u variačních autoenkodérů regularizační prvek, obdobně jako u  autoenkodérů které uvedla \autoref{chap:autoencoder}?
\end{enumerate}

\subsection{Chyba $\mathcal{D}_{KL}(q_\phi(z\mid x^{(i)})\parallel p_\theta(z))$}
\textbf{Efektivní řešitelnost} modelu variačního autoenkodéru závisí na předpokladu,
že $Q(z\mid X)$ \textbf{lze modelovat jako Gaussovu funkci se střední hodnotou $\mu(X)$ a rozptylem $\Sigma(X)$}.

Rozdělení $P(X)$ konverguje k původnímu rozdělení (které generuje vstupní data) pouze když se $\mathcal{D}_{KL}(q_\phi(z\mid x^{(i)})\parallel p_\theta(z))$ limitně blíží k nule.
Což ale nelze jednoduše zaručit. Ani za předpokladu, že $\mu(X)$ a $\Sigma(X)$ jsou vysoko-kapacitní umělé neuronové sítě
\footnote{Vysoko-kapacitní neuronové sítě, jsou sítě s vysokým počtem parametrů a vah, které jim umožňují učit se komplexním vztahům mezi daty. \cite[Kapitola 5]{Goodfellow2016}}
neplatí, že posteriorní rozdělení $P(z\mid X)$ musí být vždy Gaussovo (pro libovolnou funkci $f$ která je použita pro definování $P$).
Tedy, pro neměnné $P$ to znamená, že $\mathcal{D}_{KL}(q_\phi(z\mid x^{(i)})\parallel p_\theta(z))$ \textbf{nikdy neni rovno nule} (pokud by tato KL divergence byla rovna nule, viz \autoref{eq:elbo_kl}, znamenalo by to perfektní rekonstrukci původních dat).

Nicméně, máme-li dostatečně vysoko-kapacitní neuronové sítě, existuje mnoho funkcí $f$ (viz \autoref{sec:universal_approximation_theorem}), které zaručí, že naučený model generuje libovolné výstupní rozdělení pravděpodobnosti
\footnote{Tedy je schopen generovat vzorky \emph{dostatečně podobné} vzorkům množiny trénovacích dat, včetně vzorků, které nebyly při trénování k dispozici.}.
Libovolná z těchto funkcí maximalizuje $\log P(X)$ stejně dobře. Tedy stačí vybrat jednu z těchto funkcí, která maximalizuje $\log P(X)$ \textbf{a zároveň} zaručí, že $P (z\mid X)$ je Gaussovou funkcí pro všechna $X$.
Pokud toto platí, $\mathcal{D}_{KL}(q_\phi(z\mid x^{(i)})\parallel p_\theta(z))$, \emph{tlačí} model směrem k parametrizaci původní distribuce.

Zbývá zodpovědět, zda-li taková funkce existuje pro všechny libovolné distribuce, které bychom mohli chtít aproximovat.
Tento problém (v kontextu variačních autoenkodérů) zatím zůstává nezodpovězen. Nicméně existuje alespoň formální důkaz \textbf{nulové chyby aproximace} VAE alespoň v triviálním teoretickém scénáři na 1D problému \cite[Příloha A]{Doersch2021}.
Autoři rámce VAE se domnívají, že budoucí teoretický výzkum by měl být schopný na tomto důkazu stavět a rozšířit jej na více (složitějších) scénářů s praktickým využitím. 

\subsection{Interpretace z pohledu informační teorie}
\label{sec:vae_information_theory_interpretation}
Dalším možným (a velmi důležitým) pohledem na pravou stranu \autoref{eq:vae_elbo} je v kontextu informační teorie.
Konkrétně pomocí principu tzv. \emph{minimum description length}, který stojí za mnoha předchůdci variačního autoenkodéru, jako například Helmholtz Machines, Wake-Sleep Algoritmus, Deep Belief Nets a Boltzmann Machines.

Na $- \log P(X)$ lze pohlížet jako na celkový počet bitů, které naučený model potřebuje k sestavení daného $X$ za použití \textbf{ideálního} kódování.
Pravá strana \autoref{eq:vae_elbo} na toto pohlíží jako na dvoufázový proces pro sestavení $X$.

V první fázi použijeme \emph{nějaký počet} bitů pro sestavení $z$ \footnote{Pro připomenut KL divergenci lze uvádět v jednotkách bitů (viz \autoref{sec:kl_divergence}).}.
Konkrétně $\mathcal{D}_{KL}(q_\phi(z\mid x^{(i)})\parallel p_\theta(z))$ je očekávaná informace potřebná pro převedení vzorku z $P(z)$ (bez informace) na vzorek z $Q(z\mid X)$ (tzv. interpretace KL divergence jako informačního zisku).
Tedy měří množství nadbytečné informace které obdržíme o $X$ když obdržíme $z$ ze $Q(z\mid X)$ namísto z $P(z)$
\footnote{Pro detailněji popsanou interpretaci odkazuji na "bits back" argument z TODO CITACE }.

V druhé fázi $P(X\mid z)$ měří množství informace potřebné pro rekonstrukci $X$ ze $z$ za použití ideálního kódování.

Tedy celkový počet bitů ($- \log P(X)$) je součtem těchto dvou fází mínus trestu, který platíme za to, že $Q$ suboptimální kódováním $\mathcal{D}_{KL}(q_\phi(z\mid x^{(i)})\parallel p_\theta(z))$. 


\subsection{VAE a regularizační prvek}
\autoref{eq:vae_elbo} nabízí další pohled na $\mathcal{D}_{KL}(q_\phi(z\mid x^{(i)})\parallel p_\theta(z))$ jakožto na jakýsi regularizační prvek.
Stejně jako u řídkého autoenkodéru (viz \autoref{sec:sparse_autoencoder}), kde regularizační prvek $\Omega$ penalizuje aktivaci neuronů.
Z tohoto pohledu je zajímavé zamyslet se, zda-li u variačních autoenkodérů vůbec existuje \emph{něco jako} regularizační prvek.

U variačního autoenkodéru \textbf{regularizační prvek} obecně vůbec \textbf{neexistuje}.
Což je výhodou – jelikož to znamená o jeden prvek, který je nutné optimalizovat, méně.

Nicméně, určité modely variačního autoenkodéru se jeví, že regularizační prvek \emph{existuje}
\footnote{Nabízí se zaměnit $z \sim \mathcal{N}(0, I)$ za $z^\prime \sim \mathcal{N}(0, \Sigma * I)$, ale ukázalo se že toto naučený model nijak nezmění a ve skutečnosti vyprodukuje identickou ztrátovou funkci a model pro vzorkování $X$. \cite{Doersch2021}}.
Dobrou volbou výstupního rozdělení pravděpodobnosti pro spojitá data je $P(X\mid z) \sim \mathcal{N}(f(z), \sigma^2 * I)$ pro libovolné $\sigma$ které předem dodáme.
Tím pádem dostáváme $\log P(X\mid z) = C - \frac{1}{2} \| X - f(z) \|^2 / \sigma^2$, kde $C$ je konstanta, která nezávisí na $f$ a tak ji lze při optimalizaci zanedbat.

Pokud toto rozdělení pravděpodobnosti použijeme, ve výsledné účelové funkci se $\sigma$ objeví jako druhý (nikoliv první) prvek na pravé straně \autoref{eq:vae_elbo}.
Tedy v tomto smyslu lze zvolené $\sigma$ svou rolí považovat za jakési $\Omega$ které kontroluje váhy obou prvků pravé strany.
Nutno podotknout, že samotná existence tohoto parametru je závislá na zvoleném rozdělení pravděpodobnosti $X$ pro dané $z$.
Tedy pokud je $X$ binární, a jako výstupní model je použito Alternativní rozdělení, tento \emph{regularizační} prvek zmizí 
\footnote{I přes to existují způsoby, jakým jej lze dostat zpět do rovnice, jako například replikace dimenzí $X$. \cite{Doersch2021}}.
Což z pohledu informační teorie dává logiku: je-li $X$ binární, lze spočítat počet bitů, které jsou potřebné pro kódování $X$, a tím pádem oba prvky pravé strany \autoref{eq:vae_elbo} používají stejné jednotky.
Nicméně, pokud je $X$ spojité, každý vzorek obsahuje nekonečnou míru informace. Volba $\sigma$ pak určuje očekávanou míry přesnosti, s jakou je naučený model schopen rekonstruovat $X$ (tento kompromis je nutný, aby se míra informace vzorku stala konečná).
