\section{Testování naučeného modelu}
Pro testování modelu chceme generovat nové vzorky – toho lze dosáhnout použitím hodnot ze $z \sim \mathcal{N}$ jako vstup pro dekodér.
Jedná se tedy vlastně o odstranění enkodéru (včetně operací násobení a sčítání, které by jinak měnily rozdělení pravděpodobnosti $z$).

Schéma této jednoduché sítě je vyobrazeno v TODO FIGURE TEST-TIME NETWORK.

Chceme-li vyhodnotit pravděpodobnost vygenerování konkrétního vzorku z naučeného modelu, jedná se o efektivně neřešitelný problém (tento problém adresuje architektura tzv. Conditional variačních autoenkodérů, viz \autoref{sec:cvae}).