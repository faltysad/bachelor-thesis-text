\chapter*{Úvod}
\addcontentsline{toc}{chapter}{Úvod}
Tématem této práce je \emph{Variační autoenkodér a úlohy pozorování v latentním prostoru}.
Variační autoenkodér je model strojového učení, který představuje inovaci na poli učení bez učitele.
Jeho architektura kombinuje stochastické enkodér–dekodér moduly a hluboké učení.
Variační autoenkodér rovněž umožňuje transformovat diskrétní prostor pozorování na spojitý latentní prostor, ze kterého lze následně generovat vzorky, které nebyly součásti trénovací množiny.
Latentní prostor variačního autoenkodéru vzniká v důsledku interní reprezentace vstupních dat formou latentních proměnných.
Tyto latentní proměnné uchovávají pouze výrazné rysy vstupních dat a dochází tak k redukci jejich dimenzionality.
V důsledku toho nalézá variační autoenkodér uplatnění například v širokém spektru úloh generativního modelování.

Za průkopnickou publikaci variačního autoenkodéru je považováno dílo \textcite{Kingma2014}.
Publikované výsledky vynesly variační autoenkodér na popředí v oblasti generativního modelování.
S tím se pojí vývoj nových technik a rozšíření variačního autoenkodéru.
Ty často řeší jeho konkrétní limitaci a nebo navrhují jeho originální využití.
Mezi nejrozšířenější navazující práci patří \textcite{Sohn2015}.

Hlavním cílem této práce je \textbf{představit teorii variačního autoenkodéru a zmapovat aktuální stav poznání v oblasti}.
Dále si práce klade za cíl \textbf{představit možné aplikace variačního autoenkodéru ve vybraných problémových oblastech} a \textbf{demonstrovat využití variačního autoenkodéru formou implementace ilustrační úlohy generativního modelování obrazových dat}.
Pro naplnění cílů je čerpáno z rešerše literatury k problematice variačního autoenkodéru, která vychází především z originální publikace \textcite{Kingma2014} a navazující monografie \textcite{Kingma2019} a jimi uvedených bibliografických zdrojů.

Motivací pro zpracování cílů je aktuálnost problematiky variačního autoenkodéru, autorovo přesvědčení o správnosti využití modelu variačního autoenkodéru k tvorbě sémanticky významných latentních prostorů a univerzálnost aplikovatelnosti modelu variačního autoenkodéru na úlohy pozorování v latentním prostoru.

Primárním omezením pro naplnění cílů práce je datum originální publikace a s ním spojená omezená míra probádání problematiky variačního autoenkodéru, kdy sami autoři přiznávají řadu nedostatků bez známé příčiny \cite{Kingma2019}.
Další omezení pak představuje nízký počet netriviálních aplikací variačního autoenkodéru na úlohy pozorování v latentním prostoru.

Předpokladem pro čtenáře je znalost matematického aparátu na úrovni bakalářského kurzu matematické analýzy a statistiky.
Výhodou je pak znalost základních principů strojového učení.