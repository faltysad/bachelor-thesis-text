\chapter*{Introduction}
\addcontentsline{toc}{chapter}{Introduction}

Introduction is a compulsory part of the bachelor's / diploma thesis. The introduction is an introduction to the topic. It elaborates the chosen topic, briefly puts it into context (there may also be a description of the motivation to write the work) and answers the question why the topic was chosen. It puts the topic into context and justifies its necessity and the topicality of the solution. It contains an explicit goal of the work. The text of the thesis goal is identical with the text that is given in the bachelor's thesis assignment, ie with the text that is given in the InSIS system and which is also given in the Abstract section.

Part of the introduction is also a brief introduction to the process of processing the work (a separate part of the actual text of the work is devoted to the method of processing). The introduction may also include a description of the motivation to write the work.

The introduction to the diploma thesis must be more elaborate - this is stated in more detail in the Requirements of the diploma thesis within the Intranet for FIS students.

Here are some sample chapters that recommend how a bachelor's / master's thesis should be set. They primarily describe the use of the \LaTeX{} template, but general advice will also serve users of other systems well.